\capitulo{4}{Técnicas y herramientas}

En este apartado explicaremos las técnicas y herramientas utilizadas en nuestro proyecto.

\section{Python}

Se ha utilizado el lenguaje de programación \textit{Python}. Es un lenguaje de programación dinámicamente tipado, que soporta orientación a objetos. Posee una licencia de código abierto. Es uno de los lenguajes más utilizados para Inteligencia Artificial (IA) y  \textit{Data Science}.

\section{Anaconda}

\textit{Anaconda} es una distribución de  \textit{Python} para Cálculo numérico, análisis de datos y \textit{Machine Learning}. También es posible crear varios entornos de trabajo si estás trabajando en varios proyectos.

\section{Jupyter Notebook}

Entorno de desarrollo integrado (\textit{Integrated Development Environment} - IDE) de programación de \textit{Python} basado en \textit{IPython} de código abierto. El cuaderno Jupyter es una aplicación web que le permite crear y compartir documentos que contienen código en vivo, ecuaciones, visualizaciones y texto narrativo. Se usa para limpieza y transformación de datos, simulación numérica, modelado estadístico, visualización de datos, aprendizaje automático, etc.

\section{Spyder}

\textit{Spyder} es un IDE escrito en \textit{Python}, para \textit{Python}, y diseñado por y para científicos, ingenieros y analistas de datos. Ofrece una combinación de la funcionalidad avanzada de edición, análisis, depuración y creación de perfiles de una herramienta de desarrollo integral con la exploración de datos, ejecución interactiva, inspección profunda y capacidades de visualización de un paquete científico.

\section{Bibliotecas de Python}

\subsection{Keras}

\textit{Keras} es una biblioteca de Redes Neuronales de Código Abierto escrita en \textit{Python}. Es capaz de ejecutarse sobre \textit{TensorFlow}. Está especialmente diseñada para posibilitar la experimentación en más o menos poco tiempo con redes de Aprendizaje Profundo. Sus fuertes se centran en ser amigable para el usuario, modular y extensible.

\subsection{TensorFlow}

\textit{TensorFlow} es una librería de \textit{Python}, desarrollada por Google, para realizar cálculos numéricos mediante diagramas de flujo de datos. En vez de codificar un programa, codificaremos un grafo. Los nodos de este grafo serán operaciones matemáticas y las aristas representan los tensores (matrices de datos multidimensionales).

\subsection{NumPy}

\textit{NumPy} \cite{TyH:numpy} es una biblioteca de \textit{Python}, especializada en computación de datos. Posee gran potencial para el manejo de datos numéricos y sus operaciones, sobre todo de manera matricial, ya que contiene funciones sofisticadas y de uso simple.

\subsection{SciPy}

\textit{SciPy} \cite{TyH:scipy} es una biblioteca que contiene herramientas y algoritmos matemáticos. Su base es el objeto multidimensional de \textit{NumPy}.

\subsection{Matplotlib}

\textit{Matplolib} \cite{TyH:matplotlib} es una biblioteca que se usa para la generación y muestra de diferentes gráficos. Procesa los datos de \textit{Python} y genera diferentes salidas. Su funcionamiento es parecido a \textit{Matlab}.

\subsection{Python Imaging Library}

\textit{Python Imaging Library} (PIL) \cite{TyH:PIL} es una biblioteca gratuita que permite la edición de imágenes directamente desde \textit{Python}. Soporta una variedad de formatos, incluidos los más utilizados como GIF, JPEG y PNG. Una gran parte del código está escrito en C, por cuestiones de rendimiento.

\subsection{Scikit-image}

\textit{Scikit-image} o \textit{skimage} \cite{TyH:skimage} es un paquete de \textit{Python} dedicado al procesamiento de imágenes y al uso de matrices \textit{NumPy} nativas como objetos de imagen. Está disponible de forma gratuita y sin restricciones.

\subsection{OpenCV}

\textit{OpenCV} (\textit{Open Computer Vision}) es una biblioteca libre de visión artificial originalmente desarrollada por Intel. Se puede aplicar a detección de movimiento, reconocimiento de objetos, reconstrucción 3D a partir de imágenes y otros.

\subsection{Distutils}

\textit{Distutils} \cite{TyH:distutils} es un paquete que proporciona soporte para construir e instalar módulos adicionales en una instalación de \textit{Python}.

\section{Modelos de Python}

En este apartado describiremos dos modelos de \textit{Python} que hemos utilizado con gran frecuencia.

\subsection{Tkinter}

\textit{Tkinter} es una referencia de la biblioteca gráfica Tcl/Tk para el lenguaje de programación \textit{Python}. Es una interfaz gráfica de usuario estándar (GUI) para \textit{Python}.

\subsection{Os}

La biblioteca \textit{os} proporciona una forma sencilla de utilizar la funcionalidad del sistema operativo. Proporciona una interfaz para utilizar los comandos del mismo, independientemente de cuál sea éste.

\section{Git}

Sistema de control de versiones distribuido de código abierto. Estos sistemas son útiles ya que nos permiten funciones como volver a un punto anterior del proyecto, a parte de tener información detallada de los cambios.

\subsection{Github}

\textit{Github} es un servicio \textit{online} de código abierto que nos permite alojar nuestro repositorio del proyecto usando el control de versiones Git. Permite la integración de varias herramientas de ayuda al desarrollo, como puede ser \textit{ZenHub}.

\subsection{ZenHub}

\textit{Zenhub} es una herramienta, que se integra sobre \textit{Github}, y que nos permite llevar un mejor control del proyecto, añadiendo elementos \textit{Scrum} a nuestro repositorio \textit{Github}.

\subsection{TortoiseGit}

\textit{TortoiseGit} es una cliente de control de versiones de \textit{Git}, que nos proporciona una herramienta de escritorio para manejar nuestro repositorio en el escritorio. Nos permite utilizar todas las herramientas de \textit{Git} en una interfaz gráfica de manera sencilla.

\section{\LaTeX}

\LaTeX{} \cite{wiki:latex} es un sistema de composición de textos, orientado a la creación de documentos escritos que presenten una alta calidad tipográfica.

\subsection{Overleaf}

\textit{Overleaf} es un editor colaborativo de \LaTeX{} basado en la nube que se utiliza para escribir, editar y publicar documentos científicos. Integra variedad de herramientas para desarrollar documentos con \LaTeX{}.

\section{tmux}

\textit{tmux} \cite{tmux:repositorio} es un multiplexor de terminales, es decir, permite crear, acceder y controlar varios terminales desde una sola pantalla. \textit{tmux} puede separarse de una pantalla y continuar ejecutándose en segundo plano, pudiendo volverse a conectar posteriormente.

Hemos utilizado este programar para poder apagar el ordenador o la terminal del ordenador Alpha de la universidad mientras entrenaba. El entrenamiento podía durar unos días y de esta manera el ordenador no tenía por qué estar encendido durante todo el proceso.

\section{Técnicas}

\subsection{Patrón Singleton}

El patrón de diseño \textit{Singleton} \cite{TyH:singleton} (instancia única) está diseñado para restringir la creación de objetos pertenecientes a una clase o el valor de un tipo a un único objeto. Su intención consiste en garantizar que una clase sólo tenga una instancia y proporcionar un punto de acceso global a ella. No se encarga de la creación de objetos en sí, sino que se enfoca en la restricción en la creación de un objeto.

El uso del patrón \textit{Singleton} proporciona los siguientes beneficios:
\begin{itemize}
    \item Reduce el espacio de nombres.
    \item Controla el acceso a la instancia única.
    \item Permite el refinamiento de las operaciones y la representación.
    \item Más flexible que las operaciones de clases.
\end{itemize}

En este trabajo se ha usado el patrón \textit{Singleton} para asegurar que solo se crea una instancia del modelo de detección porque este nunca cambia desde que abrimos la aplicación y el proceso de cargar el modelo tarda bastante por lo que si sólo lo cargamos la primera vez reducimos el tiempo de detección las siguientes veces.