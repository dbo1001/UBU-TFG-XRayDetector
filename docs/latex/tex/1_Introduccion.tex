\capitulo{1}{Introducción}

El control de calidad es un aspecto importante de los procesos de fabricación. Ha aumentado el nivel de competencia en el mercado de fabricación por lo que los fabricantes deben aumentar su tasa de producción manteniendo los límites de calidad para los productos.

Las piezas de aluminio tienen actualmente un alto interés tecnológico en diversos sectores industriales (como el del automóvil) debido, entre otras, a propiedades de interés, como:

\begin{itemize}
    \item Su ligereza.
    \item Su óptima conductividad térmica y eléctrica (un conductor de aluminio pesa sólo la mitad que un conductor de cobre equivalente).
    \item Su alto porcentaje de reciclabilidad.
\end{itemize}

No obstante, durante su creación se pueden crear pequeñas burbujas o poros que son indetectables por el ojo humano aun encontrándose en la superficie. Estas burbujas pueden hacer que la pieza se rompa durante su periodo de uso, pudiendo llegar a ser un problema crítico en el caso, por ejemplo, de las piezas de automóviles ya que están sometidas a una fatiga continua.

Hay una serie de exámenes no destructivos (\textit{nondestructive examination} - NDE) que son técnicas disponibles para producir imágenes bidimensionales y tridimensionales de un objeto tales como: imágenes de rayos-X, inspección ultrasónica, inspección de partículas magnéticas u otras.

Actualmente el análisis (inspección) de estas piezas se realiza de manera visual y subjetiva por parte de personal de la propia empresa. Los empleados examinan las imágenes de rayos-X y determinan la posición de los defectos en el caso de que haya. Este proceso de análisis presenta dos grandes desventajas: el cansancio y el cambio de criterio del operario, con el paso del tiempo. Además, cada operario tiene un criterio diferente de evaluación, dificultando un seguimiento objetivo de la calidad.

La finalidad del proyecto es la de aplicar un sistema de aprendizaje basado en una red neuronal, para la identificación de estos defectos en imágenes de rayos-X tomadas principalmente de partes automotrices (ruedas de aluminio y nudillos).

El objetivo final a largo plazo sería poder comprobar la validez de una metodología automatizada que permita reducir costes y hacer el proceso más robusto y objetivo.

\section{Estructura de la memoria}

Esta memoria incluye los siguientes apartados:

\begin{itemize}
    \item \textbf{Introducción:} Breve descripción del problema a resolver y la solución propuesta. Estructura de la memoria y listado de materiales adjuntos.
    \item \textbf{Objetivos del proyecto:} Exposición de los    objetivos generales, técnicos y personales del proyecto.
    \item \textbf{Conceptos teóricos:} Breve explicación de los conceptos teóricos necesarios para la comprensión y el desarrollo del proyecto.
    \item \textbf{Técnicas y herramientas:} Presentación de las técnicas metodológicas y las herramientas de desarrollo que se han utilizado para llevar a cabo el proyecto.
    \item \textbf{Aspectos relevantes del desarrollo:} Listado o exposición de los aspectos más importantes durante el desarrollo del proyecto.
    \item \textbf{Trabajos relacionados:} Breve resumen de los trabajos y proyectos vinculados con la detección de defectos en imágenes de rayos-X y el estado del arte del proyecto.
    \item \textbf{Conclusiones y líneas de trabajo futuras:} Conclusiones obtenidas al final del proyecto y exposición de posibles mejoras o líneas de trabajo futuro.
\end{itemize}

\section{Materiales adjuntos}

Anexos aportados junto a la memoria:

\begin{itemize}
    \item \textbf{Plan del proyecto software:} Planificación temporal y estudio de viabilidad económica y legal del proyecto.
    \item \textbf{Especificación de requisitos del software:} Objetivos generales, catálogo de requisitos del sistema y especificación de requisitos funcionales y no funcionales.
    \item \textbf{Especificación de diseño:} Diseño de datos, diseño procedimental y diseño arquitectónico.
    \item \textbf{Manual del programador:} Estructura de directorios, manual del programador, compilación, instalación ejecución y pruebas (aspectos relevantes del código fuente).
    \item \textbf{Manual de usuario:} Requisitos de usuarios, instalación, manual de usuario.
\end{itemize}

Repositorio del proyecto accesible a través de internet \url{https://github.com/nuf1001/XRayDetector/} \cite{XRayDetector:repositorio}.