\capitulo{7}{Conclusiones y Líneas de trabajo futuras}

\section{Conclusión}

Respecto a los requisitos del proyecto que teníamos inicialmente considero que sí que se han llegado a cumplir en su mayor parte. Teniendo en cuenta que este proyecto ha sido un proyecto de investigación, hay objetivos que han podido cambiar. Los objetivos de realizar una investigación extensa y realizar una aplicación base de ejemplo se han cumplido.

El haber utilizado el lenguaje \textit{Python} para este proyecto ha sido una ventaja en sí. Tanto para el desarrollo como para la distribución de las clases y carpetas.

En este proyecto se han realizado una serie de experimentos sobre el análisis de imágenes y la detección de objetos en ellas bastante extensos. Teniendo una gran conjunto de datos se han podido realizar muchos experimentos.

El proyecto ha abarcado gran parte de los conocimientos adquiridos durante el grado. Además, ha requerido el aprendizaje de muchos otros como la segmentación de imágenes, OpenCV, Spyder, etc.

Se han utilizado un gran número de tecnologías nuevas. Éstas han ayudado a mejorar la calidad de los procesos intermedios que sen realizado hasta llegar al producto final. Algunas han supuesto una carga importante, como la documentación, pero todo lo aprendido será muy útil en proyectos futuros.

Gracias a la parte de investigación durante la documentación se ha aprendido a familiarizarse con la búsqueda y lectura de artículos científico.

\section{Líneas de trabajo futuras}

En esta sección describimos algunos posibles caminos diferentes de investigación y detalles a mejorar en el proyecto:

\begin{itemize}
    \item Primeramente, se sugiere que, se realicen experimentos reemplazando la capa final de la red con un \textit{Random Forest} (RF) utilizando características muestreadas de las capas de red anteriores, ya que en el artículo \cite{articulo:1} tiene un gran beneficio para la red y se generan muy buenos resultados.
    \item Se puede probar cambiar la forma de entrenar la red. En este proyecto se ha realizado con tres pasos, primero las cabeceras, a continuación la capa 4 y superior y por último todas juntas. Esto se puede modificar y encontrar una mejor forma de entrenar la red.
    \item Para terminar, se sugiere la mejora de la aplicación. Aparte de mejoras estéticas, se sugiere la inserción de más de una imagen en la aplicación teniendo un histórico de las imágenes y los defectos que se han detectado.
\end{itemize}
