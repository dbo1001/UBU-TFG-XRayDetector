\capitulo{2}{Objetivos del proyecto}

\section{Objetivos generales}

\begin{itemize}
	\item Investigar sobre técnicas del estado del arte aplicadas a la detección de defectos a través de imágenes de rayos-X con la ayuda de artículos científicos relevantes para documentar las tecnologías, herramientas y procesos utilizos.
	\item Desarollar una aplicación de escritorio que permita la cómoda utilización del software de detección de defectos en imágenes de rayos-X.
	\item Proporcionar una interpretación de los datos optenidos sencilla, facilitar al usuario su entendimiento.
	\item Analizar el rendimiento de los modelos obtenidos.
\end{itemize}

\section{Objetivos técnicos}

\begin{itemize}
    \item Aplicar el método \textbf{Mask R-CNN} para la segmentación de instancias de objetos.
	\item Desarrollar una aplicación de escritorio implementada en \textit{Python}, que refleje los resultados obtenidos para una herramienta funcional.
	\item Utilizar tkinter como una herramienta para crear la aplicación.
	\item Utilizar GitHib como herramienta para el seguimiento del proyecto junto con su extensión ZenHub.
	\item Aprender a utilizar \LaTeX{}, en especial la herramienta online Overleaf.
	\item Utilizar la metodología águil Scrum adaptada a este proyecto.
	\item Realizar test utnitarios, de sistema y de usabilidad.
\end{itemize}

\section{Objetivos personales}

\begin{itemize}
    \item Aplicar al máximo los conocimientos adquiridos por la Universidad.
    \item Aprender sobre inteligencia artificial.
    \item Aprender sobre el desarrollo de aplicaciones de escritorio.
    \item Aportar a los sectores industriales un producto que mejore la rapidez y el rendimiento dentro de las empresas.
\end{itemize}
